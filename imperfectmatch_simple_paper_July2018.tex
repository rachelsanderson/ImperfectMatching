%2multibyte Version: 5.50.0.2953 CodePage: 936
%\newtheorem{example}[theorem]{Example}
%\newtheorem{proposition}[theorem]{Proposition}
%\usepackage{cancel}
%\usepackage{xcolor}
%\usepackage{ulem}


\documentclass[12pt]{article}
%%%%%%%%%%%%%%%%%%%%%%%%%%%%%%%%%%%%%%%%%%%%%%%%%%%%%%%%%%%%%%%%%%%%%%%%%%%%%%%%%%%%%%%%%%%%%%%%%%%%%%%%%%%%%%%%%%%%%%%%%%%%%%%%%%%%%%%%%%%%%%%%%%%%%%%%%%%%%%%%%%%%%%%%%%%%%%%%%%%%%%%%%%%%%%%%%%%%%%%%%%%%%%%%%%%%%%%%%%%%%%%%%%%%%%%%%%%%%%%%%%%%%%%%%%%%
\usepackage{amsfonts}
\usepackage{amsmath}
\usepackage{amssymb}
\usepackage{epsfig}
\usepackage{float}
\usepackage[onehalfspacing]{setspace}
\usepackage{hyperref}
\usepackage{graphicx}
\usepackage{subcaption}
\usepackage{harvard,graphicx}

\setcounter{MaxMatrixCols}{10}
%TCIDATA{OutputFilter=LATEX.DLL}
%TCIDATA{Version=5.50.0.2953}
%TCIDATA{Codepage=936}
%TCIDATA{<META NAME="SaveForMode" CONTENT="1">}
%TCIDATA{BibliographyScheme=BibTeX}
%TCIDATA{Created=Monday, May 19, 2014 13:55:38}
%TCIDATA{LastRevised=Thursday, July 26, 2018 17:24:08}
%TCIDATA{<META NAME="GraphicsSave" CONTENT="32">}
%TCIDATA{<META NAME="DocumentShell" CONTENT="Standard LaTeX\Blank - Standard LaTeX Article">}
%TCIDATA{Language=American English}
%TCIDATA{CSTFile=40 LaTeX article.cst}

\newtheorem{theorem}{Theorem}
\newtheorem{acknowledgement}[theorem]{Acknowledgement}
\newtheorem{algorithm}[theorem]{Algorithm}
\newtheorem{axiom}[theorem]{Axiom}
\newtheorem{case}[theorem]{Case}
\newtheorem{claim}[theorem]{Claim}
\newtheorem{conclusion}[theorem]{Conclusion}
\newtheorem{condition}[theorem]{Condition}
\newtheorem{conjecture}[theorem]{Conjecture}
\newtheorem{corollary}[theorem]{Corollary}
\newtheorem{criterion}[theorem]{Criterion}
\newtheorem{definition}[theorem]{Definition}
\newtheorem{example}{Example}
\newtheorem{exercise}[theorem]{Exercise}
\newtheorem{lemma}[theorem]{Lemma}
\newtheorem{notation}[theorem]{Notation}
\newtheorem{problem}[theorem]{Problem}
\newtheorem{proposition}{Proposition}
\newtheorem{remark}[theorem]{Remark}
\newtheorem{solution}[theorem]{Solution}
\newtheorem{summary}[theorem]{Summary}
\newenvironment{proof}[1][Proof]{\noindent \textbf{#1.} }{\  \rule{0.5em}{0.5em}}
% Macros for Scientific Word and Scientific WorkPlace 5.5 documents saved with the LaTeX filter.
% Copyright (C) 2005 Mackichan Software, Inc.

\typeout{TCILATEX Macros for Scientific Word and Scientific WorkPlace 5.5 <06 Oct 2005>.}
\typeout{NOTICE:  This macro file is NOT proprietary and may be 
freely copied and distributed.}
%
\makeatletter

%%%%%%%%%%%%%%%%%%%%%
% pdfTeX related.
\ifx\pdfoutput\relax\let\pdfoutput=\undefined\fi
\newcount\msipdfoutput
\ifx\pdfoutput\undefined
\else
 \ifcase\pdfoutput
 \else 
    \msipdfoutput=1
    \ifx\paperwidth\undefined
    \else
      \ifdim\paperheight=0pt\relax
      \else
        \pdfpageheight\paperheight
      \fi
      \ifdim\paperwidth=0pt\relax
      \else
        \pdfpagewidth\paperwidth
      \fi
    \fi
  \fi  
\fi

%%%%%%%%%%%%%%%%%%%%%
% FMTeXButton
% This is used for putting TeXButtons in the 
% frontmatter of a document. Add a line like
% \QTagDef{FMTeXButton}{101}{} to the filter 
% section of the cst being used. Also add a
% new section containing:
%     [f_101]
%     ALIAS=FMTexButton
%     TAG_TYPE=FIELD
%     TAG_LEADIN=TeX Button:
%
% It also works to put \defs in the preamble after 
% the \input tcilatex
\def\FMTeXButton#1{#1}
%
%%%%%%%%%%%%%%%%%%%%%%
% macros for time
\newcount\@hour\newcount\@minute\chardef\@x10\chardef\@xv60
\def\tcitime{
\def\@time{%
  \@minute\time\@hour\@minute\divide\@hour\@xv
  \ifnum\@hour<\@x 0\fi\the\@hour:%
  \multiply\@hour\@xv\advance\@minute-\@hour
  \ifnum\@minute<\@x 0\fi\the\@minute
  }}%

%%%%%%%%%%%%%%%%%%%%%%
% macro for hyperref and msihyperref
%\@ifundefined{hyperref}{\def\hyperref#1#2#3#4{#2\ref{#4}#3}}{}

\def\x@hyperref#1#2#3{%
   % Turn off various catcodes before reading parameter 4
   \catcode`\~ = 12
   \catcode`\$ = 12
   \catcode`\_ = 12
   \catcode`\# = 12
   \catcode`\& = 12
   \catcode`\% = 12
   \y@hyperref{#1}{#2}{#3}%
}

\def\y@hyperref#1#2#3#4{%
   #2\ref{#4}#3
   \catcode`\~ = 13
   \catcode`\$ = 3
   \catcode`\_ = 8
   \catcode`\# = 6
   \catcode`\& = 4
   \catcode`\% = 14
}

\@ifundefined{hyperref}{\let\hyperref\x@hyperref}{}
\@ifundefined{msihyperref}{\let\msihyperref\x@hyperref}{}




% macro for external program call
\@ifundefined{qExtProgCall}{\def\qExtProgCall#1#2#3#4#5#6{\relax}}{}
%%%%%%%%%%%%%%%%%%%%%%
%
% macros for graphics
%
\def\FILENAME#1{#1}%
%
\def\QCTOpt[#1]#2{%
  \def\QCTOptB{#1}
  \def\QCTOptA{#2}
}
\def\QCTNOpt#1{%
  \def\QCTOptA{#1}
  \let\QCTOptB\empty
}
\def\Qct{%
  \@ifnextchar[{%
    \QCTOpt}{\QCTNOpt}
}
\def\QCBOpt[#1]#2{%
  \def\QCBOptB{#1}%
  \def\QCBOptA{#2}%
}
\def\QCBNOpt#1{%
  \def\QCBOptA{#1}%
  \let\QCBOptB\empty
}
\def\Qcb{%
  \@ifnextchar[{%
    \QCBOpt}{\QCBNOpt}%
}
\def\PrepCapArgs{%
  \ifx\QCBOptA\empty
    \ifx\QCTOptA\empty
      {}%
    \else
      \ifx\QCTOptB\empty
        {\QCTOptA}%
      \else
        [\QCTOptB]{\QCTOptA}%
      \fi
    \fi
  \else
    \ifx\QCBOptA\empty
      {}%
    \else
      \ifx\QCBOptB\empty
        {\QCBOptA}%
      \else
        [\QCBOptB]{\QCBOptA}%
      \fi
    \fi
  \fi
}
\newcount\GRAPHICSTYPE
%\GRAPHICSTYPE 0 is for TurboTeX
%\GRAPHICSTYPE 1 is for DVIWindo (PostScript)
%%%(removed)%\GRAPHICSTYPE 2 is for psfig (PostScript)
\GRAPHICSTYPE=\z@
\def\GRAPHICSPS#1{%
 \ifcase\GRAPHICSTYPE%\GRAPHICSTYPE=0
   \special{ps: #1}%
 \or%\GRAPHICSTYPE=1
   \special{language "PS", include "#1"}%
%%%\or%\GRAPHICSTYPE=2
%%%  #1%
 \fi
}%
%
\def\GRAPHICSHP#1{\special{include #1}}%
%
% \graffile{ body }                                  %#1
%          { contentswidth (scalar)  }               %#2
%          { contentsheight (scalar) }               %#3
%          { vertical shift when in-line (scalar) }  %#4

\def\graffile#1#2#3#4{%
%%% \ifnum\GRAPHICSTYPE=\tw@
%%%  %Following if using psfig
%%%  \@ifundefined{psfig}{\input psfig.tex}{}%
%%%  \psfig{file=#1, height=#3, width=#2}%
%%% \else
  %Following for all others
  % JCS - added BOXTHEFRAME, see below
    \bgroup
	   \@inlabelfalse
       \leavevmode
       \@ifundefined{bbl@deactivate}{\def~{\string~}}{\activesoff}%
        \raise -#4 \BOXTHEFRAME{%
           \hbox to #2{\raise #3\hbox to #2{\null #1\hfil}}}%
    \egroup
}%
%
% A box for drafts
\def\draftbox#1#2#3#4{%
 \leavevmode\raise -#4 \hbox{%
  \frame{\rlap{\protect\tiny #1}\hbox to #2%
   {\vrule height#3 width\z@ depth\z@\hfil}%
  }%
 }%
}%
%
\newcount\@msidraft
\@msidraft=\z@
\let\nographics=\@msidraft
\newif\ifwasdraft
\wasdraftfalse

%  \GRAPHIC{ body }                                  %#1
%          { draft name }                            %#2
%          { contentswidth (scalar)  }               %#3
%          { contentsheight (scalar) }               %#4
%          { vertical shift when in-line (scalar) }  %#5
\def\GRAPHIC#1#2#3#4#5{%
   \ifnum\@msidraft=\@ne\draftbox{#2}{#3}{#4}{#5}%
   \else\graffile{#1}{#3}{#4}{#5}%
   \fi
}
%
\def\addtoLaTeXparams#1{%
    \edef\LaTeXparams{\LaTeXparams #1}}%
%
% JCS -  added a switch BoxFrame that can 
% be set by including X in the frame params.
% If set a box is drawn around the frame.

\newif\ifBoxFrame \BoxFramefalse
\newif\ifOverFrame \OverFramefalse
\newif\ifUnderFrame \UnderFramefalse

\def\BOXTHEFRAME#1{%
   \hbox{%
      \ifBoxFrame
         \frame{#1}%
      \else
         {#1}%
      \fi
   }%
}


\def\doFRAMEparams#1{\BoxFramefalse\OverFramefalse\UnderFramefalse\readFRAMEparams#1\end}%
\def\readFRAMEparams#1{%
 \ifx#1\end%
  \let\next=\relax
  \else
  \ifx#1i\dispkind=\z@\fi
  \ifx#1d\dispkind=\@ne\fi
  \ifx#1f\dispkind=\tw@\fi
  \ifx#1t\addtoLaTeXparams{t}\fi
  \ifx#1b\addtoLaTeXparams{b}\fi
  \ifx#1p\addtoLaTeXparams{p}\fi
  \ifx#1h\addtoLaTeXparams{h}\fi
  \ifx#1X\BoxFrametrue\fi
  \ifx#1O\OverFrametrue\fi
  \ifx#1U\UnderFrametrue\fi
  \ifx#1w
    \ifnum\@msidraft=1\wasdrafttrue\else\wasdraftfalse\fi
    \@msidraft=\@ne
  \fi
  \let\next=\readFRAMEparams
  \fi
 \next
 }%
%
%Macro for In-line graphics object
%   \IFRAME{ contentswidth (scalar)  }               %#1
%          { contentsheight (scalar) }               %#2
%          { vertical shift when in-line (scalar) }  %#3
%          { draft name }                            %#4
%          { body }                                  %#5
%          { caption}                                %#6


\def\IFRAME#1#2#3#4#5#6{%
      \bgroup
      \let\QCTOptA\empty
      \let\QCTOptB\empty
      \let\QCBOptA\empty
      \let\QCBOptB\empty
      #6%
      \parindent=0pt
      \leftskip=0pt
      \rightskip=0pt
      \setbox0=\hbox{\QCBOptA}%
      \@tempdima=#1\relax
      \ifOverFrame
          % Do this later
          \typeout{This is not implemented yet}%
          \show\HELP
      \else
         \ifdim\wd0>\@tempdima
            \advance\@tempdima by \@tempdima
            \ifdim\wd0 >\@tempdima
               \setbox1 =\vbox{%
                  \unskip\hbox to \@tempdima{\hfill\GRAPHIC{#5}{#4}{#1}{#2}{#3}\hfill}%
                  \unskip\hbox to \@tempdima{\parbox[b]{\@tempdima}{\QCBOptA}}%
               }%
               \wd1=\@tempdima
            \else
               \textwidth=\wd0
               \setbox1 =\vbox{%
                 \noindent\hbox to \wd0{\hfill\GRAPHIC{#5}{#4}{#1}{#2}{#3}\hfill}\\%
                 \noindent\hbox{\QCBOptA}%
               }%
               \wd1=\wd0
            \fi
         \else
            \ifdim\wd0>0pt
              \hsize=\@tempdima
              \setbox1=\vbox{%
                \unskip\GRAPHIC{#5}{#4}{#1}{#2}{0pt}%
                \break
                \unskip\hbox to \@tempdima{\hfill \QCBOptA\hfill}%
              }%
              \wd1=\@tempdima
           \else
              \hsize=\@tempdima
              \setbox1=\vbox{%
                \unskip\GRAPHIC{#5}{#4}{#1}{#2}{0pt}%
              }%
              \wd1=\@tempdima
           \fi
         \fi
         \@tempdimb=\ht1
         %\advance\@tempdimb by \dp1
         \advance\@tempdimb by -#2
         \advance\@tempdimb by #3
         \leavevmode
         \raise -\@tempdimb \hbox{\box1}%
      \fi
      \egroup%
}%
%
%Macro for Display graphics object
%   \DFRAME{ contentswidth (scalar)  }               %#1
%          { contentsheight (scalar) }               %#2
%          { draft label }                           %#3
%          { name }                                  %#4
%          { caption}                                %#5
\def\DFRAME#1#2#3#4#5{%
  \vspace\topsep
  \hfil\break
  \bgroup
     \leftskip\@flushglue
	 \rightskip\@flushglue
	 \parindent\z@
	 \parfillskip\z@skip
     \let\QCTOptA\empty
     \let\QCTOptB\empty
     \let\QCBOptA\empty
     \let\QCBOptB\empty
	 \vbox\bgroup
        \ifOverFrame 
           #5\QCTOptA\par
        \fi
        \GRAPHIC{#4}{#3}{#1}{#2}{\z@}%
        \ifUnderFrame 
           \break#5\QCBOptA
        \fi
	 \egroup
  \egroup
  \vspace\topsep
  \break
}%
%
%Macro for Floating graphic object
%   \FFRAME{ framedata f|i tbph x F|T }              %#1
%          { contentswidth (scalar)  }               %#2
%          { contentsheight (scalar) }               %#3
%          { caption }                               %#4
%          { label }                                 %#5
%          { draft name }                            %#6
%          { body }                                  %#7
\def\FFRAME#1#2#3#4#5#6#7{%
 %If float.sty loaded and float option is 'h', change to 'H'  (gp) 1998/09/05
  \@ifundefined{floatstyle}
    {%floatstyle undefined (and float.sty not present), no change
     \begin{figure}[#1]%
    }
    {%floatstyle DEFINED
	 \ifx#1h%Only the h parameter, change to H
      \begin{figure}[H]%
	 \else
      \begin{figure}[#1]%
	 \fi
	}
  \let\QCTOptA\empty
  \let\QCTOptB\empty
  \let\QCBOptA\empty
  \let\QCBOptB\empty
  \ifOverFrame
    #4
    \ifx\QCTOptA\empty
    \else
      \ifx\QCTOptB\empty
        \caption{\QCTOptA}%
      \else
        \caption[\QCTOptB]{\QCTOptA}%
      \fi
    \fi
    \ifUnderFrame\else
      \label{#5}%
    \fi
  \else
    \UnderFrametrue%
  \fi
  \begin{center}\GRAPHIC{#7}{#6}{#2}{#3}{\z@}\end{center}%
  \ifUnderFrame
    #4
    \ifx\QCBOptA\empty
      \caption{}%
    \else
      \ifx\QCBOptB\empty
        \caption{\QCBOptA}%
      \else
        \caption[\QCBOptB]{\QCBOptA}%
      \fi
    \fi
    \label{#5}%
  \fi
  \end{figure}%
 }%
%
%
%    \FRAME{ framedata f|i tbph x F|T }              %#1
%          { contentswidth (scalar)  }               %#2
%          { contentsheight (scalar) }               %#3
%          { vertical shift when in-line (scalar) }  %#4
%          { caption }                               %#5
%          { label }                                 %#6
%          { name }                                  %#7
%          { body }                                  %#8
%
%    framedata is a string which can contain the following
%    characters: idftbphxFT
%    Their meaning is as follows:
%             i, d or f : in-line, display, or floating
%             t,b,p,h   : LaTeX floating placement options
%             x         : fit contents box to contents
%             F or T    : Figure or Table. 
%                         Later this can expand
%                         to a more general float class.
%
%
\newcount\dispkind%

\def\makeactives{
  \catcode`\"=\active
  \catcode`\;=\active
  \catcode`\:=\active
  \catcode`\'=\active
  \catcode`\~=\active
}
\bgroup
   \makeactives
   \gdef\activesoff{%
      \def"{\string"}%
      \def;{\string;}%
      \def:{\string:}%
      \def'{\string'}%
      \def~{\string~}%
      %\bbl@deactivate{"}%
      %\bbl@deactivate{;}%
      %\bbl@deactivate{:}%
      %\bbl@deactivate{'}%
    }
\egroup

\def\FRAME#1#2#3#4#5#6#7#8{%
 \bgroup
 \ifnum\@msidraft=\@ne
   \wasdrafttrue
 \else
   \wasdraftfalse%
 \fi
 \def\LaTeXparams{}%
 \dispkind=\z@
 \def\LaTeXparams{}%
 \doFRAMEparams{#1}%
 \ifnum\dispkind=\z@\IFRAME{#2}{#3}{#4}{#7}{#8}{#5}\else
  \ifnum\dispkind=\@ne\DFRAME{#2}{#3}{#7}{#8}{#5}\else
   \ifnum\dispkind=\tw@
    \edef\@tempa{\noexpand\FFRAME{\LaTeXparams}}%
    \@tempa{#2}{#3}{#5}{#6}{#7}{#8}%
    \fi
   \fi
  \fi
  \ifwasdraft\@msidraft=1\else\@msidraft=0\fi{}%
  \egroup
 }%
%
% This macro added to let SW gobble a parameter that
% should not be passed on and expanded. 

\def\TEXUX#1{"texux"}

%
% Macros for text attributes:
%
\def\BF#1{{\bf {#1}}}%
\def\NEG#1{\leavevmode\hbox{\rlap{\thinspace/}{$#1$}}}%
%
%%%%%%%%%%%%%%%%%%%%%%%%%%%%%%%%%%%%%%%%%%%%%%%%%%%%%%%%%%%%%%%%%%%%%%%%
%
%
% macros for user - defined functions
\def\limfunc#1{\mathop{\rm #1}}%
\def\func#1{\mathop{\rm #1}\nolimits}%
% macro for unit names
\def\unit#1{\mathord{\thinspace\rm #1}}%

%
% miscellaneous 
\long\def\QQQ#1#2{%
     \long\expandafter\def\csname#1\endcsname{#2}}%
\@ifundefined{QTP}{\def\QTP#1{}}{}
\@ifundefined{QEXCLUDE}{\def\QEXCLUDE#1{}}{}
\@ifundefined{Qlb}{\def\Qlb#1{#1}}{}
\@ifundefined{Qlt}{\def\Qlt#1{#1}}{}
\def\QWE{}%
\long\def\QQA#1#2{}%
\def\QTR#1#2{{\csname#1\endcsname {#2}}}%
  %	Add aliases for the ulem package
  \let\QQQuline\uline
  \let\QQQsout\sout
  \let\QQQuuline\uuline
  \let\QQQuwave\uwave
  \let\QQQxout\xout
\long\def\TeXButton#1#2{#2}%
\long\def\QSubDoc#1#2{#2}%
\def\EXPAND#1[#2]#3{}%
\def\NOEXPAND#1[#2]#3{}%
\def\PROTECTED{}%
\def\LaTeXparent#1{}%
\def\ChildStyles#1{}%
\def\ChildDefaults#1{}%
\def\QTagDef#1#2#3{}%

% Constructs added with Scientific Notebook
\@ifundefined{correctchoice}{\def\correctchoice{\relax}}{}
\@ifundefined{HTML}{\def\HTML#1{\relax}}{}
\@ifundefined{TCIIcon}{\def\TCIIcon#1#2#3#4{\relax}}{}
\if@compatibility
  \typeout{Not defining UNICODE  U or CustomNote commands for LaTeX 2.09.}
\else
  \providecommand{\UNICODE}[2][]{\protect\rule{.1in}{.1in}}
  \providecommand{\U}[1]{\protect\rule{.1in}{.1in}}
  \providecommand{\CustomNote}[3][]{\marginpar{#3}}
\fi

\@ifundefined{lambdabar}{
      \def\lambdabar{\errmessage{You have used the lambdabar symbol. 
                      This is available for typesetting only in RevTeX styles.}}
   }{}

%
% Macros for style editor docs
\@ifundefined{StyleEditBeginDoc}{\def\StyleEditBeginDoc{\relax}}{}
%
% Macros for footnotes
\def\QQfnmark#1{\footnotemark}
\def\QQfntext#1#2{\addtocounter{footnote}{#1}\footnotetext{#2}}
%
% Macros for indexing.
%
\@ifundefined{TCIMAKEINDEX}{}{\makeindex}%
%
% Attempts to avoid problems with other styles
\@ifundefined{abstract}{%
 \def\abstract{%
  \if@twocolumn
   \section*{Abstract (Not appropriate in this style!)}%
   \else \small 
   \begin{center}{\bf Abstract\vspace{-.5em}\vspace{\z@}}\end{center}%
   \quotation 
   \fi
  }%
 }{%
 }%
\@ifundefined{endabstract}{\def\endabstract
  {\if@twocolumn\else\endquotation\fi}}{}%
\@ifundefined{maketitle}{\def\maketitle#1{}}{}%
\@ifundefined{affiliation}{\def\affiliation#1{}}{}%
\@ifundefined{proof}{\def\proof{\noindent{\bfseries Proof. }}}{}%
\@ifundefined{endproof}{\def\endproof{\mbox{\ \rule{.1in}{.1in}}}}{}%
\@ifundefined{newfield}{\def\newfield#1#2{}}{}%
\@ifundefined{chapter}{\def\chapter#1{\par(Chapter head:)#1\par }%
 \newcount\c@chapter}{}%
\@ifundefined{part}{\def\part#1{\par(Part head:)#1\par }}{}%
\@ifundefined{section}{\def\section#1{\par(Section head:)#1\par }}{}%
\@ifundefined{subsection}{\def\subsection#1%
 {\par(Subsection head:)#1\par }}{}%
\@ifundefined{subsubsection}{\def\subsubsection#1%
 {\par(Subsubsection head:)#1\par }}{}%
\@ifundefined{paragraph}{\def\paragraph#1%
 {\par(Subsubsubsection head:)#1\par }}{}%
\@ifundefined{subparagraph}{\def\subparagraph#1%
 {\par(Subsubsubsubsection head:)#1\par }}{}%
%%%%%%%%%%%%%%%%%%%%%%%%%%%%%%%%%%%%%%%%%%%%%%%%%%%%%%%%%%%%%%%%%%%%%%%%
% These symbols are not recognized by LaTeX
\@ifundefined{therefore}{\def\therefore{}}{}%
\@ifundefined{backepsilon}{\def\backepsilon{}}{}%
\@ifundefined{yen}{\def\yen{\hbox{\rm\rlap=Y}}}{}%
\@ifundefined{registered}{%
   \def\registered{\relax\ifmmode{}\r@gistered
                    \else$\m@th\r@gistered$\fi}%
 \def\r@gistered{^{\ooalign
  {\hfil\raise.07ex\hbox{$\scriptstyle\rm\text{R}$}\hfil\crcr
  \mathhexbox20D}}}}{}%
\@ifundefined{Eth}{\def\Eth{}}{}%
\@ifundefined{eth}{\def\eth{}}{}%
\@ifundefined{Thorn}{\def\Thorn{}}{}%
\@ifundefined{thorn}{\def\thorn{}}{}%
% A macro to allow any symbol that requires math to appear in text
\def\TEXTsymbol#1{\mbox{$#1$}}%
\@ifundefined{degree}{\def\degree{{}^{\circ}}}{}%
%
% macros for T3TeX files
\newdimen\theight
\@ifundefined{Column}{\def\Column{%
 \vadjust{\setbox\z@=\hbox{\scriptsize\quad\quad tcol}%
  \theight=\ht\z@\advance\theight by \dp\z@\advance\theight by \lineskip
  \kern -\theight \vbox to \theight{%
   \rightline{\rlap{\box\z@}}%
   \vss
   }%
  }%
 }}{}%
%
\@ifundefined{qed}{\def\qed{%
 \ifhmode\unskip\nobreak\fi\ifmmode\ifinner\else\hskip5\p@\fi\fi
 \hbox{\hskip5\p@\vrule width4\p@ height6\p@ depth1.5\p@\hskip\p@}%
 }}{}%
%
\@ifundefined{cents}{\def\cents{\hbox{\rm\rlap c/}}}{}%
\@ifundefined{tciLaplace}{\def\tciLaplace{\ensuremath{\mathcal{L}}}}{}%
\@ifundefined{tciFourier}{\def\tciFourier{\ensuremath{\mathcal{F}}}}{}%
\@ifundefined{textcurrency}{\def\textcurrency{\hbox{\rm\rlap xo}}}{}%
\@ifundefined{texteuro}{\def\texteuro{\hbox{\rm\rlap C=}}}{}%
\@ifundefined{euro}{\def\euro{\hbox{\rm\rlap C=}}}{}%
\@ifundefined{textfranc}{\def\textfranc{\hbox{\rm\rlap-F}}}{}%
\@ifundefined{textlira}{\def\textlira{\hbox{\rm\rlap L=}}}{}%
\@ifundefined{textpeseta}{\def\textpeseta{\hbox{\rm P\negthinspace s}}}{}%
%
\@ifundefined{miss}{\def\miss{\hbox{\vrule height2\p@ width 2\p@ depth\z@}}}{}%
%
\@ifundefined{vvert}{\def\vvert{\Vert}}{}%  %always translated to \left| or \right|
%
\@ifundefined{tcol}{\def\tcol#1{{\baselineskip=6\p@ \vcenter{#1}} \Column}}{}%
%
\@ifundefined{dB}{\def\dB{\hbox{{}}}}{}%        %dummy entry in column 
\@ifundefined{mB}{\def\mB#1{\hbox{$#1$}}}{}%   %column entry
\@ifundefined{nB}{\def\nB#1{\hbox{#1}}}{}%     %column entry (not math)
%
\@ifundefined{note}{\def\note{$^{\dag}}}{}%
%
\def\newfmtname{LaTeX2e}
% No longer load latexsym.  This is now handled by SWP, which uses amsfonts if necessary
%
\ifx\fmtname\newfmtname
  \DeclareOldFontCommand{\rm}{\normalfont\rmfamily}{\mathrm}
  \DeclareOldFontCommand{\sf}{\normalfont\sffamily}{\mathsf}
  \DeclareOldFontCommand{\tt}{\normalfont\ttfamily}{\mathtt}
  \DeclareOldFontCommand{\bf}{\normalfont\bfseries}{\mathbf}
  \DeclareOldFontCommand{\it}{\normalfont\itshape}{\mathit}
  \DeclareOldFontCommand{\sl}{\normalfont\slshape}{\@nomath\sl}
  \DeclareOldFontCommand{\sc}{\normalfont\scshape}{\@nomath\sc}
\fi

%
% Greek bold macros
% Redefine all of the math symbols 
% which might be bolded	 - there are 
% probably others to add to this list

\def\alpha{{\Greekmath 010B}}%
\def\beta{{\Greekmath 010C}}%
\def\gamma{{\Greekmath 010D}}%
\def\delta{{\Greekmath 010E}}%
\def\epsilon{{\Greekmath 010F}}%
\def\zeta{{\Greekmath 0110}}%
\def\eta{{\Greekmath 0111}}%
\def\theta{{\Greekmath 0112}}%
\def\iota{{\Greekmath 0113}}%
\def\kappa{{\Greekmath 0114}}%
\def\lambda{{\Greekmath 0115}}%
\def\mu{{\Greekmath 0116}}%
\def\nu{{\Greekmath 0117}}%
\def\xi{{\Greekmath 0118}}%
\def\pi{{\Greekmath 0119}}%
\def\rho{{\Greekmath 011A}}%
\def\sigma{{\Greekmath 011B}}%
\def\tau{{\Greekmath 011C}}%
\def\upsilon{{\Greekmath 011D}}%
\def\phi{{\Greekmath 011E}}%
\def\chi{{\Greekmath 011F}}%
\def\psi{{\Greekmath 0120}}%
\def\omega{{\Greekmath 0121}}%
\def\varepsilon{{\Greekmath 0122}}%
\def\vartheta{{\Greekmath 0123}}%
\def\varpi{{\Greekmath 0124}}%
\def\varrho{{\Greekmath 0125}}%
\def\varsigma{{\Greekmath 0126}}%
\def\varphi{{\Greekmath 0127}}%

\def\nabla{{\Greekmath 0272}}
\def\FindBoldGroup{%
   {\setbox0=\hbox{$\mathbf{x\global\edef\theboldgroup{\the\mathgroup}}$}}%
}

\def\Greekmath#1#2#3#4{%
    \if@compatibility
        \ifnum\mathgroup=\symbold
           \mathchoice{\mbox{\boldmath$\displaystyle\mathchar"#1#2#3#4$}}%
                      {\mbox{\boldmath$\textstyle\mathchar"#1#2#3#4$}}%
                      {\mbox{\boldmath$\scriptstyle\mathchar"#1#2#3#4$}}%
                      {\mbox{\boldmath$\scriptscriptstyle\mathchar"#1#2#3#4$}}%
        \else
           \mathchar"#1#2#3#4% 
        \fi 
    \else 
        \FindBoldGroup
        \ifnum\mathgroup=\theboldgroup % For 2e
           \mathchoice{\mbox{\boldmath$\displaystyle\mathchar"#1#2#3#4$}}%
                      {\mbox{\boldmath$\textstyle\mathchar"#1#2#3#4$}}%
                      {\mbox{\boldmath$\scriptstyle\mathchar"#1#2#3#4$}}%
                      {\mbox{\boldmath$\scriptscriptstyle\mathchar"#1#2#3#4$}}%
        \else
           \mathchar"#1#2#3#4% 
        \fi     	    
	  \fi}

\newif\ifGreekBold  \GreekBoldfalse
\let\SAVEPBF=\pbf
\def\pbf{\GreekBoldtrue\SAVEPBF}%
%

\@ifundefined{theorem}{\newtheorem{theorem}{Theorem}}{}
\@ifundefined{lemma}{\newtheorem{lemma}[theorem]{Lemma}}{}
\@ifundefined{corollary}{\newtheorem{corollary}[theorem]{Corollary}}{}
\@ifundefined{conjecture}{\newtheorem{conjecture}[theorem]{Conjecture}}{}
\@ifundefined{proposition}{\newtheorem{proposition}[theorem]{Proposition}}{}
\@ifundefined{axiom}{\newtheorem{axiom}{Axiom}}{}
\@ifundefined{remark}{\newtheorem{remark}{Remark}}{}
\@ifundefined{example}{\newtheorem{example}{Example}}{}
\@ifundefined{exercise}{\newtheorem{exercise}{Exercise}}{}
\@ifundefined{definition}{\newtheorem{definition}{Definition}}{}


\@ifundefined{mathletters}{%
  %\def\theequation{\arabic{equation}}
  \newcounter{equationnumber}  
  \def\mathletters{%
     \addtocounter{equation}{1}
     \edef\@currentlabel{\theequation}%
     \setcounter{equationnumber}{\c@equation}
     \setcounter{equation}{0}%
     \edef\theequation{\@currentlabel\noexpand\alph{equation}}%
  }
  \def\endmathletters{%
     \setcounter{equation}{\value{equationnumber}}%
  }
}{}

%Logos
\@ifundefined{BibTeX}{%
    \def\BibTeX{{\rm B\kern-.05em{\sc i\kern-.025em b}\kern-.08em
                 T\kern-.1667em\lower.7ex\hbox{E}\kern-.125emX}}}{}%
\@ifundefined{AmS}%
    {\def\AmS{{\protect\usefont{OMS}{cmsy}{m}{n}%
                A\kern-.1667em\lower.5ex\hbox{M}\kern-.125emS}}}{}%
\@ifundefined{AmSTeX}{\def\AmSTeX{\protect\AmS-\protect\TeX\@}}{}%
%

% This macro is a fix to eqnarray
\def\@@eqncr{\let\@tempa\relax
    \ifcase\@eqcnt \def\@tempa{& & &}\or \def\@tempa{& &}%
      \else \def\@tempa{&}\fi
     \@tempa
     \if@eqnsw
        \iftag@
           \@taggnum
        \else
           \@eqnnum\stepcounter{equation}%
        \fi
     \fi
     \global\tag@false
     \global\@eqnswtrue
     \global\@eqcnt\z@\cr}


\def\TCItag{\@ifnextchar*{\@TCItagstar}{\@TCItag}}
\def\@TCItag#1{%
    \global\tag@true
    \global\def\@taggnum{(#1)}%
    \global\def\@currentlabel{#1}}
\def\@TCItagstar*#1{%
    \global\tag@true
    \global\def\@taggnum{#1}%
    \global\def\@currentlabel{#1}}
%
%%%%%%%%%%%%%%%%%%%%%%%%%%%%%%%%%%%%%%%%%%%%%%%%%%%%%%%%%%%%%%%%%%%%%
%
\def\QATOP#1#2{{#1 \atop #2}}%
\def\QTATOP#1#2{{\textstyle {#1 \atop #2}}}%
\def\QDATOP#1#2{{\displaystyle {#1 \atop #2}}}%
\def\QABOVE#1#2#3{{#2 \above#1 #3}}%
\def\QTABOVE#1#2#3{{\textstyle {#2 \above#1 #3}}}%
\def\QDABOVE#1#2#3{{\displaystyle {#2 \above#1 #3}}}%
\def\QOVERD#1#2#3#4{{#3 \overwithdelims#1#2 #4}}%
\def\QTOVERD#1#2#3#4{{\textstyle {#3 \overwithdelims#1#2 #4}}}%
\def\QDOVERD#1#2#3#4{{\displaystyle {#3 \overwithdelims#1#2 #4}}}%
\def\QATOPD#1#2#3#4{{#3 \atopwithdelims#1#2 #4}}%
\def\QTATOPD#1#2#3#4{{\textstyle {#3 \atopwithdelims#1#2 #4}}}%
\def\QDATOPD#1#2#3#4{{\displaystyle {#3 \atopwithdelims#1#2 #4}}}%
\def\QABOVED#1#2#3#4#5{{#4 \abovewithdelims#1#2#3 #5}}%
\def\QTABOVED#1#2#3#4#5{{\textstyle 
   {#4 \abovewithdelims#1#2#3 #5}}}%
\def\QDABOVED#1#2#3#4#5{{\displaystyle 
   {#4 \abovewithdelims#1#2#3 #5}}}%
%
% Macros for text size operators:
%

\def\tint{\msi@int\textstyle\int}%
\def\tiint{\msi@int\textstyle\iint}%
\def\tiiint{\msi@int\textstyle\iiint}%
\def\tiiiint{\msi@int\textstyle\iiiint}%
\def\tidotsint{\msi@int\textstyle\idotsint}%
\def\toint{\msi@int\textstyle\oint}%


\def\tsum{\mathop{\textstyle \sum }}%
\def\tprod{\mathop{\textstyle \prod }}%
\def\tbigcap{\mathop{\textstyle \bigcap }}%
\def\tbigwedge{\mathop{\textstyle \bigwedge }}%
\def\tbigoplus{\mathop{\textstyle \bigoplus }}%
\def\tbigodot{\mathop{\textstyle \bigodot }}%
\def\tbigsqcup{\mathop{\textstyle \bigsqcup }}%
\def\tcoprod{\mathop{\textstyle \coprod }}%
\def\tbigcup{\mathop{\textstyle \bigcup }}%
\def\tbigvee{\mathop{\textstyle \bigvee }}%
\def\tbigotimes{\mathop{\textstyle \bigotimes }}%
\def\tbiguplus{\mathop{\textstyle \biguplus }}%
%
%
%Macros for display size operators:
%

\newtoks\temptoksa
\newtoks\temptoksb
\newtoks\temptoksc


\def\msi@int#1#2{%
 \def\@temp{{#1#2\the\temptoksc_{\the\temptoksa}^{\the\temptoksb}}}%   
 \futurelet\@nextcs
 \@int
}

\def\@int{%
   \ifx\@nextcs\limits
      \typeout{Found limits}%
      \temptoksc={\limits}%
	  \let\@next\@intgobble%
   \else\ifx\@nextcs\nolimits
      \typeout{Found nolimits}%
      \temptoksc={\nolimits}%
	  \let\@next\@intgobble%
   \else
      \typeout{Did not find limits or no limits}%
      \temptoksc={}%
      \let\@next\msi@limits%
   \fi\fi
   \@next   
}%

\def\@intgobble#1{%
   \typeout{arg is #1}%
   \msi@limits
}


\def\msi@limits{%
   \temptoksa={}%
   \temptoksb={}%
   \@ifnextchar_{\@limitsa}{\@limitsb}%
}

\def\@limitsa_#1{%
   \temptoksa={#1}%
   \@ifnextchar^{\@limitsc}{\@temp}%
}

\def\@limitsb{%
   \@ifnextchar^{\@limitsc}{\@temp}%
}

\def\@limitsc^#1{%
   \temptoksb={#1}%
   \@ifnextchar_{\@limitsd}{\@temp}%   
}

\def\@limitsd_#1{%
   \temptoksa={#1}%
   \@temp
}



\def\dint{\msi@int\displaystyle\int}%
\def\diint{\msi@int\displaystyle\iint}%
\def\diiint{\msi@int\displaystyle\iiint}%
\def\diiiint{\msi@int\displaystyle\iiiint}%
\def\didotsint{\msi@int\displaystyle\idotsint}%
\def\doint{\msi@int\displaystyle\oint}%

\def\dsum{\mathop{\displaystyle \sum }}%
\def\dprod{\mathop{\displaystyle \prod }}%
\def\dbigcap{\mathop{\displaystyle \bigcap }}%
\def\dbigwedge{\mathop{\displaystyle \bigwedge }}%
\def\dbigoplus{\mathop{\displaystyle \bigoplus }}%
\def\dbigodot{\mathop{\displaystyle \bigodot }}%
\def\dbigsqcup{\mathop{\displaystyle \bigsqcup }}%
\def\dcoprod{\mathop{\displaystyle \coprod }}%
\def\dbigcup{\mathop{\displaystyle \bigcup }}%
\def\dbigvee{\mathop{\displaystyle \bigvee }}%
\def\dbigotimes{\mathop{\displaystyle \bigotimes }}%
\def\dbiguplus{\mathop{\displaystyle \biguplus }}%

\if@compatibility\else
  % Always load amsmath in LaTeX2e mode
  \RequirePackage{amsmath}
\fi

\def\ExitTCILatex{\makeatother\endinput}

\bgroup
\ifx\ds@amstex\relax
   \message{amstex already loaded}\aftergroup\ExitTCILatex
\else
   \@ifpackageloaded{amsmath}%
      {\if@compatibility\message{amsmath already loaded}\fi\aftergroup\ExitTCILatex}
      {}
   \@ifpackageloaded{amstex}%
      {\if@compatibility\message{amstex already loaded}\fi\aftergroup\ExitTCILatex}
      {}
   \@ifpackageloaded{amsgen}%
      {\if@compatibility\message{amsgen already loaded}\fi\aftergroup\ExitTCILatex}
      {}
\fi
\egroup

%Exit if any of the AMS macros are already loaded.
%This is always the case for LaTeX2e mode.


%%%%%%%%%%%%%%%%%%%%%%%%%%%%%%%%%%%%%%%%%%%%%%%%%%%%%%%%%%%%%%%%%%%%%%%%%%
% NOTE: The rest of this file is read only if in LaTeX 2.09 compatibility
% mode. This section is used to define AMS-like constructs in the
% event they have not been defined.
%%%%%%%%%%%%%%%%%%%%%%%%%%%%%%%%%%%%%%%%%%%%%%%%%%%%%%%%%%%%%%%%%%%%%%%%%%
\typeout{TCILATEX defining AMS-like constructs in LaTeX 2.09 COMPATIBILITY MODE}
%%%%%%%%%%%%%%%%%%%%%%%%%%%%%%%%%%%%%%%%%%%%%%%%%%%%%%%%%%%%%%%%%%%%%%%%
%  Macros to define some AMS LaTeX constructs when 
%  AMS LaTeX has not been loaded
% 
% These macros are copied from the AMS-TeX package for doing
% multiple integrals.
%
\let\DOTSI\relax
\def\RIfM@{\relax\ifmmode}%
\def\FN@{\futurelet\next}%
\newcount\intno@
\def\iint{\DOTSI\intno@\tw@\FN@\ints@}%
\def\iiint{\DOTSI\intno@\thr@@\FN@\ints@}%
\def\iiiint{\DOTSI\intno@4 \FN@\ints@}%
\def\idotsint{\DOTSI\intno@\z@\FN@\ints@}%
\def\ints@{\findlimits@\ints@@}%
\newif\iflimtoken@
\newif\iflimits@
\def\findlimits@{\limtoken@true\ifx\next\limits\limits@true
 \else\ifx\next\nolimits\limits@false\else
 \limtoken@false\ifx\ilimits@\nolimits\limits@false\else
 \ifinner\limits@false\else\limits@true\fi\fi\fi\fi}%
\def\multint@{\int\ifnum\intno@=\z@\intdots@                          %1
 \else\intkern@\fi                                                    %2
 \ifnum\intno@>\tw@\int\intkern@\fi                                   %3
 \ifnum\intno@>\thr@@\int\intkern@\fi                                 %4
 \int}%                                                               %5
\def\multintlimits@{\intop\ifnum\intno@=\z@\intdots@\else\intkern@\fi
 \ifnum\intno@>\tw@\intop\intkern@\fi
 \ifnum\intno@>\thr@@\intop\intkern@\fi\intop}%
\def\intic@{%
    \mathchoice{\hskip.5em}{\hskip.4em}{\hskip.4em}{\hskip.4em}}%
\def\negintic@{\mathchoice
 {\hskip-.5em}{\hskip-.4em}{\hskip-.4em}{\hskip-.4em}}%
\def\ints@@{\iflimtoken@                                              %1
 \def\ints@@@{\iflimits@\negintic@
   \mathop{\intic@\multintlimits@}\limits                             %2
  \else\multint@\nolimits\fi                                          %3
  \eat@}%                                                             %4
 \else                                                                %5
 \def\ints@@@{\iflimits@\negintic@
  \mathop{\intic@\multintlimits@}\limits\else
  \multint@\nolimits\fi}\fi\ints@@@}%
\def\intkern@{\mathchoice{\!\!\!}{\!\!}{\!\!}{\!\!}}%
\def\plaincdots@{\mathinner{\cdotp\cdotp\cdotp}}%
\def\intdots@{\mathchoice{\plaincdots@}%
 {{\cdotp}\mkern1.5mu{\cdotp}\mkern1.5mu{\cdotp}}%
 {{\cdotp}\mkern1mu{\cdotp}\mkern1mu{\cdotp}}%
 {{\cdotp}\mkern1mu{\cdotp}\mkern1mu{\cdotp}}}%
%
%
%  These macros are for doing the AMS \text{} construct
%
\def\RIfM@{\relax\protect\ifmmode}
\def\text{\RIfM@\expandafter\text@\else\expandafter\mbox\fi}
\let\nfss@text\text
\def\text@#1{\mathchoice
   {\textdef@\displaystyle\f@size{#1}}%
   {\textdef@\textstyle\tf@size{\firstchoice@false #1}}%
   {\textdef@\textstyle\sf@size{\firstchoice@false #1}}%
   {\textdef@\textstyle \ssf@size{\firstchoice@false #1}}%
   \glb@settings}

\def\textdef@#1#2#3{\hbox{{%
                    \everymath{#1}%
                    \let\f@size#2\selectfont
                    #3}}}
\newif\iffirstchoice@
\firstchoice@true
%
%These are the AMS constructs for multiline limits.
%
\def\Let@{\relax\iffalse{\fi\let\\=\cr\iffalse}\fi}%
\def\vspace@{\def\vspace##1{\crcr\noalign{\vskip##1\relax}}}%
\def\multilimits@{\bgroup\vspace@\Let@
 \baselineskip\fontdimen10 \scriptfont\tw@
 \advance\baselineskip\fontdimen12 \scriptfont\tw@
 \lineskip\thr@@\fontdimen8 \scriptfont\thr@@
 \lineskiplimit\lineskip
 \vbox\bgroup\ialign\bgroup\hfil$\m@th\scriptstyle{##}$\hfil\crcr}%
\def\Sb{_\multilimits@}%
\def\endSb{\crcr\egroup\egroup\egroup}%
\def\Sp{^\multilimits@}%
\let\endSp\endSb
%
%
%These are AMS constructs for horizontal arrows
%
\newdimen\ex@
\ex@.2326ex
\def\rightarrowfill@#1{$#1\m@th\mathord-\mkern-6mu\cleaders
 \hbox{$#1\mkern-2mu\mathord-\mkern-2mu$}\hfill
 \mkern-6mu\mathord\rightarrow$}%
\def\leftarrowfill@#1{$#1\m@th\mathord\leftarrow\mkern-6mu\cleaders
 \hbox{$#1\mkern-2mu\mathord-\mkern-2mu$}\hfill\mkern-6mu\mathord-$}%
\def\leftrightarrowfill@#1{$#1\m@th\mathord\leftarrow
\mkern-6mu\cleaders
 \hbox{$#1\mkern-2mu\mathord-\mkern-2mu$}\hfill
 \mkern-6mu\mathord\rightarrow$}%
\def\overrightarrow{\mathpalette\overrightarrow@}%
\def\overrightarrow@#1#2{\vbox{\ialign{##\crcr\rightarrowfill@#1\crcr
 \noalign{\kern-\ex@\nointerlineskip}$\m@th\hfil#1#2\hfil$\crcr}}}%
\let\overarrow\overrightarrow
\def\overleftarrow{\mathpalette\overleftarrow@}%
\def\overleftarrow@#1#2{\vbox{\ialign{##\crcr\leftarrowfill@#1\crcr
 \noalign{\kern-\ex@\nointerlineskip}$\m@th\hfil#1#2\hfil$\crcr}}}%
\def\overleftrightarrow{\mathpalette\overleftrightarrow@}%
\def\overleftrightarrow@#1#2{\vbox{\ialign{##\crcr
   \leftrightarrowfill@#1\crcr
 \noalign{\kern-\ex@\nointerlineskip}$\m@th\hfil#1#2\hfil$\crcr}}}%
\def\underrightarrow{\mathpalette\underrightarrow@}%
\def\underrightarrow@#1#2{\vtop{\ialign{##\crcr$\m@th\hfil#1#2\hfil
  $\crcr\noalign{\nointerlineskip}\rightarrowfill@#1\crcr}}}%
\let\underarrow\underrightarrow
\def\underleftarrow{\mathpalette\underleftarrow@}%
\def\underleftarrow@#1#2{\vtop{\ialign{##\crcr$\m@th\hfil#1#2\hfil
  $\crcr\noalign{\nointerlineskip}\leftarrowfill@#1\crcr}}}%
\def\underleftrightarrow{\mathpalette\underleftrightarrow@}%
\def\underleftrightarrow@#1#2{\vtop{\ialign{##\crcr$\m@th
  \hfil#1#2\hfil$\crcr
 \noalign{\nointerlineskip}\leftrightarrowfill@#1\crcr}}}%
%%%%%%%%%%%%%%%%%%%%%

\def\qopnamewl@#1{\mathop{\operator@font#1}\nlimits@}
\let\nlimits@\displaylimits
\def\setboxz@h{\setbox\z@\hbox}


\def\varlim@#1#2{\mathop{\vtop{\ialign{##\crcr
 \hfil$#1\m@th\operator@font lim$\hfil\crcr
 \noalign{\nointerlineskip}#2#1\crcr
 \noalign{\nointerlineskip\kern-\ex@}\crcr}}}}

 \def\rightarrowfill@#1{\m@th\setboxz@h{$#1-$}\ht\z@\z@
  $#1\copy\z@\mkern-6mu\cleaders
  \hbox{$#1\mkern-2mu\box\z@\mkern-2mu$}\hfill
  \mkern-6mu\mathord\rightarrow$}
\def\leftarrowfill@#1{\m@th\setboxz@h{$#1-$}\ht\z@\z@
  $#1\mathord\leftarrow\mkern-6mu\cleaders
  \hbox{$#1\mkern-2mu\copy\z@\mkern-2mu$}\hfill
  \mkern-6mu\box\z@$}


\def\projlim{\qopnamewl@{proj\,lim}}
\def\injlim{\qopnamewl@{inj\,lim}}
\def\varinjlim{\mathpalette\varlim@\rightarrowfill@}
\def\varprojlim{\mathpalette\varlim@\leftarrowfill@}
\def\varliminf{\mathpalette\varliminf@{}}
\def\varliminf@#1{\mathop{\underline{\vrule\@depth.2\ex@\@width\z@
   \hbox{$#1\m@th\operator@font lim$}}}}
\def\varlimsup{\mathpalette\varlimsup@{}}
\def\varlimsup@#1{\mathop{\overline
  {\hbox{$#1\m@th\operator@font lim$}}}}

%
%Companion to stackrel
\def\stackunder#1#2{\mathrel{\mathop{#2}\limits_{#1}}}%
%
%
% These are AMS environments that will be defined to
% be verbatims if amstex has not actually been 
% loaded
%
%
\begingroup \catcode `|=0 \catcode `[= 1
\catcode`]=2 \catcode `\{=12 \catcode `\}=12
\catcode`\\=12 
|gdef|@alignverbatim#1\end{align}[#1|end[align]]
|gdef|@salignverbatim#1\end{align*}[#1|end[align*]]

|gdef|@alignatverbatim#1\end{alignat}[#1|end[alignat]]
|gdef|@salignatverbatim#1\end{alignat*}[#1|end[alignat*]]

|gdef|@xalignatverbatim#1\end{xalignat}[#1|end[xalignat]]
|gdef|@sxalignatverbatim#1\end{xalignat*}[#1|end[xalignat*]]

|gdef|@gatherverbatim#1\end{gather}[#1|end[gather]]
|gdef|@sgatherverbatim#1\end{gather*}[#1|end[gather*]]

|gdef|@gatherverbatim#1\end{gather}[#1|end[gather]]
|gdef|@sgatherverbatim#1\end{gather*}[#1|end[gather*]]


|gdef|@multilineverbatim#1\end{multiline}[#1|end[multiline]]
|gdef|@smultilineverbatim#1\end{multiline*}[#1|end[multiline*]]

|gdef|@arraxverbatim#1\end{arrax}[#1|end[arrax]]
|gdef|@sarraxverbatim#1\end{arrax*}[#1|end[arrax*]]

|gdef|@tabulaxverbatim#1\end{tabulax}[#1|end[tabulax]]
|gdef|@stabulaxverbatim#1\end{tabulax*}[#1|end[tabulax*]]


|endgroup
  

  
\def\align{\@verbatim \frenchspacing\@vobeyspaces \@alignverbatim
You are using the "align" environment in a style in which it is not defined.}
\let\endalign=\endtrivlist
 
\@namedef{align*}{\@verbatim\@salignverbatim
You are using the "align*" environment in a style in which it is not defined.}
\expandafter\let\csname endalign*\endcsname =\endtrivlist




\def\alignat{\@verbatim \frenchspacing\@vobeyspaces \@alignatverbatim
You are using the "alignat" environment in a style in which it is not defined.}
\let\endalignat=\endtrivlist
 
\@namedef{alignat*}{\@verbatim\@salignatverbatim
You are using the "alignat*" environment in a style in which it is not defined.}
\expandafter\let\csname endalignat*\endcsname =\endtrivlist




\def\xalignat{\@verbatim \frenchspacing\@vobeyspaces \@xalignatverbatim
You are using the "xalignat" environment in a style in which it is not defined.}
\let\endxalignat=\endtrivlist
 
\@namedef{xalignat*}{\@verbatim\@sxalignatverbatim
You are using the "xalignat*" environment in a style in which it is not defined.}
\expandafter\let\csname endxalignat*\endcsname =\endtrivlist




\def\gather{\@verbatim \frenchspacing\@vobeyspaces \@gatherverbatim
You are using the "gather" environment in a style in which it is not defined.}
\let\endgather=\endtrivlist
 
\@namedef{gather*}{\@verbatim\@sgatherverbatim
You are using the "gather*" environment in a style in which it is not defined.}
\expandafter\let\csname endgather*\endcsname =\endtrivlist


\def\multiline{\@verbatim \frenchspacing\@vobeyspaces \@multilineverbatim
You are using the "multiline" environment in a style in which it is not defined.}
\let\endmultiline=\endtrivlist
 
\@namedef{multiline*}{\@verbatim\@smultilineverbatim
You are using the "multiline*" environment in a style in which it is not defined.}
\expandafter\let\csname endmultiline*\endcsname =\endtrivlist


\def\arrax{\@verbatim \frenchspacing\@vobeyspaces \@arraxverbatim
You are using a type of "array" construct that is only allowed in AmS-LaTeX.}
\let\endarrax=\endtrivlist

\def\tabulax{\@verbatim \frenchspacing\@vobeyspaces \@tabulaxverbatim
You are using a type of "tabular" construct that is only allowed in AmS-LaTeX.}
\let\endtabulax=\endtrivlist

 
\@namedef{arrax*}{\@verbatim\@sarraxverbatim
You are using a type of "array*" construct that is only allowed in AmS-LaTeX.}
\expandafter\let\csname endarrax*\endcsname =\endtrivlist

\@namedef{tabulax*}{\@verbatim\@stabulaxverbatim
You are using a type of "tabular*" construct that is only allowed in AmS-LaTeX.}
\expandafter\let\csname endtabulax*\endcsname =\endtrivlist

% macro to simulate ams tag construct


% This macro is a fix to the equation environment
 \def\endequation{%
     \ifmmode\ifinner % FLEQN hack
      \iftag@
        \addtocounter{equation}{-1} % undo the increment made in the begin part
        $\hfil
           \displaywidth\linewidth\@taggnum\egroup \endtrivlist
        \global\tag@false
        \global\@ignoretrue   
      \else
        $\hfil
           \displaywidth\linewidth\@eqnnum\egroup \endtrivlist
        \global\tag@false
        \global\@ignoretrue 
      \fi
     \else   
      \iftag@
        \addtocounter{equation}{-1} % undo the increment made in the begin part
        \eqno \hbox{\@taggnum}
        \global\tag@false%
        $$\global\@ignoretrue
      \else
        \eqno \hbox{\@eqnnum}% $$ BRACE MATCHING HACK
        $$\global\@ignoretrue
      \fi
     \fi\fi
 } 

 \newif\iftag@ \tag@false
 
 \def\TCItag{\@ifnextchar*{\@TCItagstar}{\@TCItag}}
 \def\@TCItag#1{%
     \global\tag@true
     \global\def\@taggnum{(#1)}%
     \global\def\@currentlabel{#1}}
 \def\@TCItagstar*#1{%
     \global\tag@true
     \global\def\@taggnum{#1}%
     \global\def\@currentlabel{#1}}

  \@ifundefined{tag}{
     \def\tag{\@ifnextchar*{\@tagstar}{\@tag}}
     \def\@tag#1{%
         \global\tag@true
         \global\def\@taggnum{(#1)}}
     \def\@tagstar*#1{%
         \global\tag@true
         \global\def\@taggnum{#1}}
  }{}

\def\tfrac#1#2{{\textstyle {#1 \over #2}}}%
\def\dfrac#1#2{{\displaystyle {#1 \over #2}}}%
\def\binom#1#2{{#1 \choose #2}}%
\def\tbinom#1#2{{\textstyle {#1 \choose #2}}}%
\def\dbinom#1#2{{\displaystyle {#1 \choose #2}}}%

% Do not add anything to the end of this file.  
% The last section of the file is loaded only if 
% amstex has not been.
\makeatother
\endinput

\setlength{\topmargin}{-0.4in} \setlength{\textheight}{8.9in}
\setlength{\evensidemargin}{0.25in}
\setlength{\oddsidemargin}{0.25in} \setlength{\textwidth}{6.0in}
\setlength{\parskip}{0ex} \setlength{\footnotesep}{12pt}
\setcounter{page}{1}
\renewcommand{\baselinestretch}{1.50}
\renewcommand{\arraystretch}{1}
\renewcommand{\cite}{\citeasnoun}
\newcommand*\dispfrac[2]{\frac{\displaystyle #1}{\displaystyle #2}}
\newcommand\redsout{\bgroup\markoverwith{\textcolor{red}{\rule[0.5ex]{2pt}{2.0pt}}}\ULon}
\begin{document}

\author{Rachel L. Anderson\thanks{%
Mailing Address: Department of Economics, Julis Romo Rabinowitz Building,
Princeton, NJ 08544. Email: rachelsa@Princeton.edu.} \and Bo E. Honor\'{e}\thanks{%
Mailing Address: Department of Economics, Julis Romo Rabinowitz Building,
Princeton, NJ 08544. Email: honore@Princeton.edu.} \and Adriana Lleras-Muney%
\thanks{%
Mailing Address: UCLA.} \and At Most one More}
\title{Estimation and inference using imperfectly matched data}
\maketitle

\begin{abstract}
\singlespacing
This paper studies estimation and inference in standard econometric models
that use matched data sets with multiple matches of the dependent variable.
We show that it is straightforward to use multiple matches provided that the
true match is included among the potential matches. On the other hand,
identification is generally not possible if the true match is not included.
We also investigate the possibility of using information about the quality
of a match. The main result here is that if teh probability of a correct
match is only approximate then using the match quality will lead to
inconsistent estimators although we also illustrate that bias-variance
tradeoffs canjustify using approximate match information informations.
\end{abstract}

%\maketitle

\section{Introduction}

The recent availability of large administrative data sets has increased the
use of matched data in recent years in economic applications (Chetty 2012
not in references). When matching data sets, there will be three possible
outcomes, the consequences of which we need to account for in the analysis.
Some observations will not be matched to an outcome and thus will result in
missing data (recipients for whom no outcomes can be found). Among those
that are matched, it is not always possible to find a unique match, thus for
some individuals we will have multiple matches/outcomes. Finally there might
measurement error in matching: even when find a unique match per individual,
it might not correspond to the correct match.

Many studies by economic historians and historical demographers have
employed record linkage techniques. For example, \cite{Abramitzky2012} link
Norwegian-born from the Norwegian census of 1865 to the 1900 U.S. Census of
Population and the 1900 Norwegian census, finding only 26\% of those
observed in the base year. \cite{FerrieRolf2011} link males born 1895-1900
from the 5\% IPUMS sample of the 1900 U.S. Census of Population to the
Social Security Death Master File, and successfully link 29\% of those
sought. \cite{CollinsWanamaker2013} are able to link 21\% of black,
Southern-born males from the 1\% IPUMS sample of the 1910 U.S. Census of
Population to the 1930 U.S. Census of Population.

There are several prominent recent examples that match administrative data
bases to do program evaluation. The Moving to Opportunity experiment,
offering vouchers to poor families to move to low-poverty areas, was
evaluated by matching data on participants to many administrative data sets
including state UI data, state AFDC/TANF/Food stamps data, juvenile and
criminal justice administrative data, National Student Clearinghouse data on
college going, school data among others (\cite{Kling2007}, \cite{Kling2005}%
). \cite{Chettyteachers2013} look at the long-run effects of smaller classes
and better teachers, by matching the original Tennessee Project STAR
experimental data to later IRS administrative tax records of the children
when they are adults. \cite{Chettykindergarten2011} match 2.5M NYC public
school records to IRS data to look at the long-run impacts of teacher value
added. \cite{DobbieFryer2011} evaluate the effect of the Harlem Children's
Zone by matching the HCZ data to the National Student Clearinghouse college
enrollment data.

There are also many publicly available and commonly used datasets that are
constructed by matching administrative data sets, for instance the Linked
Birth and Infant Death data (matches birth certificates and death
certificates from population databases), the IPUMS Linked Representative
Samples, 1850-1930 (matches individuals from census to census to create a
panel data), the National Longitudinal Mortality Survey (matches CPS data to
death certificates from the National Death Index, or NDI), the National
Health Interview survey Linked Mortality File (NHIS survey linked to NDI),
the General Social Survey-National Death Index (GSS-NDI) and National Health
and Nutrition Examination Survey Linked Mortality Files (I, II, and III).
Even in these high quality data sets, containing social security numbers,
there are a non-trivial number of multiple matches and measurement error.%
\footnote{%
need a reference}

The standard approach in the existing literature that uses matched data
consists in dropping any observation without a match or with multiple
matches, and to treat unique matches as correct. Inference and estimation
then proceeds by treating the data as if it came from a single source. In
this paper we investigate how to best use matched data by a-incorporating
multiple matches and b-allowing for measurement error in matching.

\section{General problem}

We are interested in estimating a model of the general form:
\begin{equation}
E\left[ m\left( y_{i},x_{i},z_{i};\theta _{0}\right) \right] =0
\label{Moment}
\end{equation}%
where $y_{i}$, $x_{i}$ and $z_{i}$ are vectors or scalars of data for an
individual $i$. We will typically think of $y_{i}$ as the dependent variable
(such as earnings, disability or age at death), and ($x_{i}$, $z_{i}$) as a
set of individual-level covariates or instruments. The function $m\left(
y_{i},x_{i},z_{i};\theta _{0}\right) $ is known. $\theta _{0}$ is the
parameter of interest.

The complication of the estimation problem we consider arises because the
variables $y_{i}$, $x_{i}$ and $z_{i}$ are not observed in a single data
set. Instead we have access to two different data sets. The first (the $y$%
--dataset) contains $y_{i}$ and another vector of characteristics, $w_{i}$.
The other (the $x$--dataset) contains $x_{i}$, $z_{i}$ and $w_{i}$. The
vector $w_{i}$ can be used to match observations across data sets but not
always uniquely -- in other words $w_{i}$ is not a unique person identifier.
Specifically, for each observation, $i$, in the $x$--dataset there will be $%
L_{i}$ \textquotedblleft matching\textquotedblright\ observations in the $y$%
--dataset with identical $w_{i}$. In the applications that we have in mind,
the $y$--dataset will in principle be the population. This implies that one
of the $L_{i}$ matched $y$'s will correspond to observation $i$ and the
remaining $\left( L_{i}-1\right) $ $y$'s are mismatches. This will be our
leading case, but we will also discuss extension to the case where it is
possible that none of the matches is correct.

For example \cite{Finkelstein2012} match individuals in the experiment to
hospital discharge records, credit reports and mortality to look at the
impact of offering Medicaid on health care use, bankrupcy, and health. In
this case the $x$--dataset contains then list of individuals that were
randomly assigned to Medicaid. The $y$--dataset is one of the administrative
data sets, for instance the hospital discharge records. The data sets are
matched based on name and date of birth which are known in both data ($w_{i}$%
). In this example (and in all others cited above), the administrative data
contain the information for the population. In principle all individuals in
the $x$--dataset should be found in the $y$--dataset, but they are not,
because name and date of birth do not uniquely identify individuals, and
because of record errors (mispelled last names for instance).

\section{GMM}

Consider an individual from the $x$--dataset with $L_{i}$ matches in the $y$%
--dataset. The data for that observation is $\left( x_{i},z_{i},\left\{
y_{i\ell }\right\} _{\ell =1}^{L_{i}},w_{i}\right) $. We start by
considering the case where the correct match is in the set of matches and
where every element in that set is equally likely to be the correct match.
For instance consider the case where we observe an individual
\textquotedblleft Joe Smith\textquotedblright\ in the $x$--dataset born in
January 3 1984. In the $y$--dataset we find two male individuals born on the
same date with names \textquotedblleft Joe A. Smith\textquotedblright\ and
\textquotedblleft Joseph B. Smith\textquotedblright . In the absence of
additional information we assume that each of the two matches is equally
likely to be the correct match. For a fixed matching criteria, the number of
matches $L_{i}$ is a random variable.

Since we are matching observations on the basis of $w_{i}$, we will assume
that the $y$--dataset and the $x$--dataset are random samples conditional on
$w_{i}$. We also need to need to clarify that the moment condition (\ref%
{Moment}) is assumed to hold for the distribution generating the $x$%
--dataset.

Let
\begin{equation*}
g_{1}\left( x,z,L,w;\theta \right) =E\left[ \left. m\left(
y_{j},x,z,L;\theta \right) \right\vert w_{j}=w\right]
\end{equation*}%
then

We then have%
\begin{equation*}
E\left[ \left. \sum_{\ell =1}^{L_{i}}m\left( y_{i\ell },x_{i},z_{i};\theta
\right) \right\vert x_{i},z_{i},L_{i}\right] =E\left[ \left. m\left(
y_{i},x_{i},z_{i};\theta \right) \right\vert x_{i},z_{i},L_{i}\right]
+\left( L_{i}-1\right) g_{1}\left( x_{i},z_{i},L_{i},w_{i};\theta \right)
\end{equation*}%
and%
\begin{eqnarray*}
&&E\left[ \left. m\left( y_{i\ell },x_{i},z_{i};\theta \right) \right\vert
z_{i}=z,x_{i}=x,w_{i}=w\right]  \\
&=&\frac{1}{L_{i}}E\left[ \left. m\left( y_{i},x_{i},z_{i};\theta \right)
\right\vert z_{i}=z,x_{i}=x,w_{i}=w\right]  \\
&&+\frac{L_{i}-1}{L_{i}}E\left[ \left. m\left( y_{j},x,z;\theta \right)
\right\vert w_{j}=w\right]
\end{eqnarray*}%
or%
\begin{eqnarray*}
E\left[ \left. m\left( y_{i},x_{i},z_{i};\theta \right) \right\vert
z_{i}=z,x_{i}=x,w_{i}=w\right]  &=&\sum_{\ell }E\left[ \left. m\left(
y_{i\ell },x_{i},z_{i};\theta \right) \right\vert z_{i}=z,x_{i}=x,w_{i}=w%
\right]  \\
&&-\left( L_{i}-1\right) E\left[ \left. m\left( y_{j},x,z;\theta \right)
\right\vert w_{j}=w\right]
\end{eqnarray*}

Let
\begin{equation*}
g\left( x,z,w\right) =E\left[ \left. m\left( y_{j},x,z;\theta \right)
\right\vert w_{j}=w\right]
\end{equation*}%
then
\begin{equation}
E\left[ m\left( y_{i},x_{i},z_{i};\theta \right) \right] =\sum_{\ell }E\left[
m\left( y_{i\ell },x_{i},z_{i};\theta \right) \right] -E\left[ \left(
L_{i}-1\right) g\left( x_{i},z_{i},w_{i}\right) \right]
\label{main equation}
\end{equation}%
where the expectations are with respect to the distributions generating the $%
x$--data set. The right-hand side of (\ref{main equation}) can be estimated
given our setup, and the left-hand side identifies the parameter of interest
via the moment condition (\ref{Moment}).

[remark: this case covers multiple xs in addition to many ys--compare to WP
by Mahajan or Poirer)

\subsection{Linear Instrumental Variables Estimation\label{Lin IV}}

Consider the text--book linear instrumental variables model
\begin{equation*}
y_{i}=x_{i}^{\prime }\beta +\varepsilon _{i}\qquad E\left[ z_{i}\varepsilon
_{i}\right] =0
\end{equation*}%
or%
\begin{equation}
E\left[ z_{i}\left( y_{i}-x_{i}^{\prime }\beta \right) \right] =0
\label{moment condition}
\end{equation}%
or%
\begin{equation*}
E\left[ z_{i}y_{i}\right] =E\left[ z_{i}x_{i}^{\prime }\right] \beta
\end{equation*}%
In this case (\ref{main equation}) becomes%
\begin{eqnarray*}
E\left[ z_{i}\left( y_{i}-x_{i}^{\prime }b\right) \right] &=&\sum_{\ell }E%
\left[ z_{i}\left( y_{i\ell }-x_{i}^{\prime }b\right) \right] -E\left[
\left( L_{i}-1\right) z_{i}\left( g\left( w_{i}\right) -x_{i}^{\prime
}b\right) \right] \\
&=&E\left[ z_{i}\left( \left( \sum_{\ell }y_{i\ell }-\left( L_{i}-1\right)
g\left( w_{i}\right) \right) -x_{i}^{\prime }b\right) \right]
\end{eqnarray*}%
where $g\left( w_{i}\right) =E\left[ \left. y_{i}\right\vert w_{i}\right] $.
In other words, $\beta $ satisfies the moment condition
\begin{equation}
E\left[ z_{i}\left( \left( \sum_{\ell }y_{i\ell }-\left( L_{i}-1\right)
g\left( w_{i}\right) \right) -x_{i}^{\prime }\beta \right) \right] =0
\label{momcon1}
\end{equation}

When the model is just--identified this gives the explicit expression for $%
\beta $%
\begin{equation}
\beta =E\left[ z_{i}x_{i}^{\prime }\right] ^{-1}E\left[ z_{i}\left(
\sum_{\ell }y_{i\ell }-\left( L_{i}-1\right) g\left( w_{i}\right) \right) %
\right]  \label{est1}
\end{equation}%
When $z_{i}=x_{i}$ and $L_{i}=1$, this is the OLS estimator in a regression
of $y_{i}$ on $x_{i}$.

When the model is over--identified ($\dim \left( z_{i}\right) =m>k=\dim
\left( x_{i}\right) $), we have%
\begin{equation}
\beta =\left( PE\left[ z_{i}x_{i}^{\prime }\right] \right) ^{-1}PE\left[
z_{i}\left( \sum_{\ell }y_{i\ell }-\left( L_{i}-1\right) g\left(
w_{i}\right) \right) \right]  \label{est2}
\end{equation}%
for any $k\times m$-matrix such that $E\left[ Pz_{i}x_{i}^{\prime }\right]
^{-1}$exists. The choice of $P$ is equivalent to the choice of weighting
matrix in GMM. For example $P=E\left[ x_{i}z_{i}^{\prime }\right] $ $E\left[
z_{i}z_{i}^{\prime }\right] ^{-1}$ yields the usual 2SLS\ estimator when $%
L_{i}=1$ for all $i$.

We now turn to the problem of converting (\ref{est1}) and (\ref{est2}) into
estimators. The first complication in this is that $g\left( \cdot \right) $
must be estimated. In many applications, the $y$--dataset will be much, much
larger than the $x$--dataset and it is then reasonable to treat $g\left(
\cdot \right) $ as if it is known. In other cases, we will explicitly think
of $g\left( \cdot \right) $ as an object to be estimated. The second
complication is that (\ref{moment condition}) is often thought of as an
implication of the conditional moment condition
\begin{equation}
E\left[ \left. \left( y_{i}-x_{i}^{\prime }\beta \right) \right\vert z_{i}%
\right] =0.  \label{cond moment}
\end{equation}%
In this case there is room for improving efficiency by weighing different
observations differently when forming sample analogues to expectations.

To fix ideas, consider first the case where $g\left( \cdot \right) $ is
known and the starting point is (\ref{moment condition}). In that case the
optimal GMM\ estimator is
\begin{equation*}
\left( \left( \dsum x_{i}z_{i}^{\prime }\right) W\left( \dsum
x_{i}z_{i}^{\prime }\right) ^{\prime }\right) ^{-1}\left( \dsum
x_{i}z_{i}^{\prime }\right) W\left( \dsum x_{i}y_{i}\right)
\end{equation*}%
where%
\begin{equation*}
W=E\left[ \left( \left( \sum_{\ell }y_{i\ell }-\left( L_{i}-1\right) g\left(
w_{i}\right) \right) -x_{i}^{\prime }b\right) ^{2}z_{i}z_{i}^{\prime }\right]
^{-1}
\end{equation*}%
Let $\nu _{i\ell }=y_{i\ell }-g\left( w_{i}\right) $ then $W$ can be written
as%
\begin{equation*}
W=E\left[ \left( \varepsilon _{i}+\dsum \nu _{i\ell }\right)
^{2}z_{i}z_{i}^{\prime }\right] ^{-1}
\end{equation*}%
where the sum if over the $L_{i}-1$ incorrect matches.

If $E\left[ \left. \nu _{i\ell }^{2}\right\vert z_{i},L_{i}\right] =\sigma
_{\nu }^{2}$ and $E\left[ \left. \varepsilon _{i}^{2}\right\vert z_{i},L_{i}%
\right] =\sigma _{\varepsilon }^{2}$ then (assuming independence)
\begin{equation*}
W=E\left[ \left( \sigma _{\varepsilon }^{2}+\left( L_{i}-1\right) \sigma
_{\nu }^{2}\right) E\left[ \left. z_{i}z_{i}^{\prime }\right\vert L_{i}%
\right] \right] ^{-1}
\end{equation*}%
If in addition $E\left[ \left. \varepsilon _{i}\right\vert z_{i},L_{i}\right]
=0$ then
\begin{eqnarray*}
\left( \sum_{\ell }y_{i\ell }-\left( L_{i}-1\right) g\left( w_{i}\right)
\right) &=&x_{i}^{\prime }b+\varepsilon _{i}+\dsum \nu _{i\ell } \\
&=&x_{i}^{\prime }b+u_{i}
\end{eqnarray*}%
with $E\left[ \left. u_{i}\right\vert z_{i},L_{i}\right] =0$ and $V\left[
\left. u_{i}\right\vert z_{i},L_{i}\right] =\left( \sigma _{\varepsilon
}^{2}+\left( L_{i}-1\right) \sigma _{\nu }^{2}\right) $ and the efficient
estimator (check and reference) is... weighted 2sls... When $z_{i}=x_{i}$
the optimal estimator of $\beta $ then is the weighted least squares
estimator
\begin{equation*}
\left( \sum_{i}\frac{1}{\left( \sigma _{\varepsilon }^{2}+\left(
L_{i}-1\right) \sigma _{\nu }^{2}\right) }x_{i}x_{i}^{\prime }\right)
^{-1}\left( \sum_{i}\frac{1}{\left( \sigma _{\varepsilon }^{2}+\left(
L_{i}-1\right) \sigma _{\nu }^{2}\right) }x_{i}\left( \sum_{\ell
=1}^{L_{i}}y_{i\ell }-\left( L_{i}-1\right) g\left( w_{i}\right) \right)
\right)
\end{equation*}%
where $\sigma _{\varepsilon }^{2}$ and $\sigma _{\nu }^{2}$ can be replaced
by consistent estimators.

We next consider the case where $g\left( \cdot \right) $ is parameterized,
so we write $g\left( w_{i}\right) =g\left( w_{i};\alpha \right) $ and
estimate $\alpha $ by some standard estimator that can be written as a
solution to a moment condition. For simplicity, we first assume that it is
estimated from the sample of matches $\left\{ \left\{ y_{i\ell }\right\}
_{\ell =1}^{L_{i}},w_{i}\right\} _{i=1}^{n}$ , but it could also be
estimated from a larger sample.

There are then two cases to consider. One where we first estimate $\alpha $
and then subsequently $\beta $, and one where $\alpha $ and $\beta $ are
potentially estimated jointly. In the former case, we stack the sample
moment conditions determining $\widehat{\alpha }$ and the moment conditions
that determine $\widehat{\beta }$:%
\begin{equation*}
\left(
\begin{array}{c}
\frac{1}{n}\sum_{i}\left( \sum_{\ell }\rho \left( y_{i\ell },w_{i},\widehat{%
\alpha }\right) \right) \\
\frac{1}{n}\sum_{i}Pz_{i}\left( \left( \sum_{\ell }y_{i\ell }-\left(
L_{i}-1\right) g\left( w_{i};\widehat{\alpha }\right) \right) -x_{i}^{\prime
}\widehat{\beta }\right)%
\end{array}%
\right) =\left(
\begin{array}{c}
0 \\
0%
\end{array}%
\right)
\end{equation*}%
These deliver the asymptotic distribution of $\left( \widehat{\alpha },%
\widehat{\beta }\right) $. If one allows $\alpha $ and $\beta $ to be
estimated jointly, then one can simply consider a GMM estimator of $\left(
\alpha ,\beta \right) $ based on the moment conditions
\begin{equation*}
\left(
\begin{array}{c}
E\left[ \sum_{\ell }\rho \left( y_{i\ell },w_{i},\alpha \right) \right] \\
E\left[ z_{i}\left( \left( \sum_{\ell }y_{i\ell }-\left( L_{i}-1\right)
g\left( w_{i};\alpha \right) \right) -x_{i}^{\prime }\beta \right) \right]%
\end{array}%
\right) =\left(
\begin{array}{c}
0 \\
0%
\end{array}%
\right)
\end{equation*}

When $g\left( \cdot \right) $ is estimated nonparametrically for a (much)
larger sample than the $x$--data set, the standard erros for $\widehat{\beta
}$ can be calculated as in \cite{ChenHongTamer2005}. See also \cite%
{ChenHongTarozzi2008}

\subsection{Maximum Likelihood Estimation}

We next turn to estimation of parametric nonlinear model which would
typically be estimated by maximum likelihood estimation if there were no
multiple matches.

One way to approach this is to think of the first order condition for
maximum likelihood as a moment condition and then proceed as above.
Alternatively, one might think of maximum likelihood estimation in the
presence of multiple matches.

Using the same notation as earlier, assume that an observation consists of $%
\left( x_{i},\left\{ y_{i\ell }\right\} _{\ell =1}^{L_{i}},w_{i}\right) $.
For $\ell =1,...,L$, there is probability $\frac{1}{L_{i}}$ that $y_{i\ell }$
is drawn from the parametric model, $f\left( \left. y;\theta \right\vert
x_{i}\right) $; in this case, $y_{ik}$ is drawn from some pre-estimated
\textquotedblleft reduced form\textquotedblright\ $g\left( \left.
y\right\vert w_{i}\right) $ for $k\neq \ell .$

This gives the likelihood function
\begin{align*}
& \sum_{\ell }^{{}}\left\{ \frac{1}{L_{i}}f\left( \left. y_{i\ell };\theta
\right\vert x_{i}\right) \dprod\limits_{k\neq \ell }^{{}}\ g\left( \left.
y_{ik}\right\vert w_{i}\right) \right\} \\
& =\dprod\limits_{k}^{{}}\ g\left( \left. y_{ik}\right\vert w_{i}\right)
\left( \sum_{\ell }^{{}}\frac{1}{L_{i}}\frac{f\left( \left. y_{i\ell
};\theta \right\vert x_{i}\right) }{g\left( \left. y_{i\ell }\right\vert
w_{i}\right) }\right)
\end{align*}%
So except for a constant, the (pseudo) log--likelihood function is
\begin{equation}
\sum_{i}\ln \left( \sum_{\ell }^{{}}\frac{1}{L_{i}}\frac{f\left( \left.
y_{i\ell };\theta \right\vert x_{i}\right) }{g\left( \left. y_{i\ell
}\right\vert w_{i}\right) }\right)  \label{like-logit}
\end{equation}%
The trick is coming up with a good $g$. This will have to be
application--specific.

The asymptotic properties of the estimator defined by maximizing (\ref%
{like-logit}) will again depend on how one thinks of $g$. In some cases, the
\textquotedblleft y\textquotedblright -dataset will be so large relative to
the \textquotedblleft x\textquotedblright --dataset that it is reasonable to
consider $g$ known. In that case the maximizer of (\ref{like-logit}) is a
standard maximum likelihood estimator. In other cases, one will want to
account for the fact that $g$ has been estimated (parametrically or
nonparametrically).

\subsubsection{Relationship to methods of moments (optional)}

The first order condition for maximizing (\ref{like-logit}) is%
\begin{equation}
\sum_{i}\left( \sum_{\ell }^{{}}\frac{f\left( \left. y_{i\ell };\theta
\right\vert x_{i}\right) }{g\left( \left. y_{i\ell }\right\vert w_{i}\right)
}\right) ^{-1}\sum_{\ell }^{{}}\frac{f^{\prime }\left( \left. y_{i\ell
};\theta \right\vert x_{i}\right) }{g\left( \left. y_{i\ell }\right\vert
w_{i}\right) }=0.  \label{FOC-lik}
\end{equation}%
By comparison, the estimator defined by (\ref{main equation}) with $m()$ the
derivative of $f$ with respect to $\theta $ would solve%
\begin{equation}
\sum_{i}\left( \sum_{\ell }f^{\prime }\left( \left. y_{i\ell };\theta
\right\vert x_{i}\right) -\left( L_{i}-1\right) E\left[ \left. f^{\prime
}\left( \left. y;\theta \right\vert x_{i}\right) \right\vert w\right]
\right) =0
\end{equation}%
It seems that this is different from (\ref{FOC-lik}).

\section{Possible Generalizations}

\subsection{Using information about the match quality}

When presented with multiple matches, the researcher will often have
information about which of the matches is most likely to be correct. In this
section, we argue that using this information can lead to biases unless it
is possible to consistently estimate the probability that each match is
correct. We make this pooint by considering a very simple example.

Suppose we want to estimate a mean, $\mu =E\left[ X\right] $. For each $i$,
we have two observations, $X_{1i}$ and $X_{2i}$. One is drawn from the
correct distribution which has mean $\mu $ and variance $\sigma ^{2}$ and
one is drawn from a known incorrect distribution with \textsl{known} mean $%
\kappa $ and variance $\omega ^{2}$.

Suppose that the probability that the first is drawn from the correct
correct distribution is $\pi $. Then%
\begin{eqnarray*}
E\left[ X_{1i}\right] &=&\pi \mu +\left( 1-\pi \right) \kappa \\
E\left[ X_{2i}\right] &=&\pi \kappa +\left( 1-\pi \right) \mu
\end{eqnarray*}%
so
\begin{equation*}
E\left[ X_{1i}+X_{2i}\right] -\kappa =\pi \left( \mu +\kappa \right) +\left(
1-\pi \right) \left( \kappa +\mu \right) -\kappa =\mu
\end{equation*}%
It therefore follows that
\begin{equation*}
\frac{1}{n}\dsum\limits_{i=1}^{n}\left( X_{1i}+X_{2i}\right) -\kappa
\end{equation*}%
is a consistent estimator of $\mu $.

More generally consider an estimator of the form
\begin{equation}
\widehat{\mu }=\frac{a_{1}}{n}\dsum\limits_{i=1}^{n}X_{1i}+\frac{a_{2}}{n}%
\dsum\limits_{i=1}^{n}X_{2i}-a_{3}\kappa  \label{muhat}
\end{equation}%
Its mean would be
\begin{eqnarray}
&&E\left[ \widehat{\mu }\right] =\pi \left( a_{1}\mu +a_{2}\kappa \right)
+\left( 1-\pi \right) \left( a_{1}\kappa +a_{2}\mu \right) -a_{3}\kappa
\label{e0} \\
&=&\left( \pi a_{1}+\left( 1-\pi \right) a_{2}\right) \mu +\left( \pi
a_{2}+\left( 1-\pi \right) a_{1}-a_{3}\right) \kappa  \notag
\end{eqnarray}%
For unbiasedness, we then need%
\begin{equation}
\left( \pi a_{1}+\left( 1-\pi \right) a_{2}\right) =1  \label{e1}
\end{equation}%
or%
\begin{equation*}
a_{2}=\frac{1-\pi a_{1}}{1-\pi }=\frac{1}{1-\pi }-\frac{\pi }{1-\pi }a_{1}
\end{equation*}%
and%
\begin{equation}
\left( \pi a_{2}+\left( 1-\pi \right) a_{1}-a_{3}\right) =0  \label{e2}
\end{equation}%
The only way to do this without knowing $\pi $ is to set $a_{1}=a_{2}$. But
in that case (\ref{e2})\ inplies that $a_{1}=a_{2}=1$.

If we know $\pi $ then (\ref{e1} and (\ref{e2}) can be solved for $a_{2}$
and $a_{3}$ as a function of $a_{1}$:

\begin{center}
[here insert numerical example]
\end{center}

REMARK:\ By Oct 13, have a numerical example.

Also think about this:\ if we postulate a $\pi _{1}$ , calculate the bias
and variance of various estimators when $\pi _{1}\neq \pi .$

If possible do this for OLS\ as well

\subsection{Mean-Variance Trade-Off}

Simple formulas and numerical illustration

Perhaps also try to do explicit calculations for OLS

\section{Allowing for the correct match to not be included}

\subsection{This section will argue that identification is not possible in
this case}

We next consider the case where there is some probability that none of the
observations is drawn from the distribution of interest. In the motivation
setup, this corresponds to the case where none of the matches is the correct
one.

Suppose again that we want to estimate a mean, $\mu =E\left[ X\right] $. For
each $i$, we have two observations, $X_{1i}$ and $X_{2i}$. With probability $%
\pi _{j}$, $X_{ji}$ is drawn from the correct distribution which has mean $%
\mu $ and variance $\sigma ^{2}$ and the other is drawn from a known
incorrect distribution with \textsl{known} mean $\kappa $ and variance $%
\omega ^{2}$. With probability $1-\pi _{1}-\pi _{2}$, $X_{1i}$ and $X_{2i}$
are drawn independently from the known incorrect distribution with \textsl{%
known} mean $\kappa $ and variance $\omega ^{2}$.

Consider an estimator of the (natural) form
\begin{equation}
\widehat{\mu }=\frac{a_{1}}{n}\dsum\limits_{i=1}^{n}X_{1i}+\frac{a_{2}}{n}%
\dsum\limits_{i=1}^{n}X_{2i}-a_{3}\kappa  \label{muhat2}
\end{equation}%
Its mean would be
\begin{eqnarray}
&&a_{1}\left( \pi _{1}\mu +\left( 1-\pi _{1}\right) \kappa \right)
+a_{2}\left( \pi _{2}\mu +\left( 1-\pi _{2}\right) \kappa \right)
-a_{3}\kappa  \notag \\
&=&\left( a_{1}\pi _{1}+a_{2}\pi _{2}\right) \mu +\left( a_{1}\left( 1-\pi
_{1}\right) +a_{2}\left( 1-\pi _{2}\right) -a_{3}\right) \kappa
\label{mean2}
\end{eqnarray}%
Without knowledge of $\pi _{1}$ and $\pi _{2}$ is is impossible to choose $%
a_{1}$ and $a_{2}$ such that this is always equal to $\mu $. This would be
true even if we knew that $\pi _{1}=\pi _{2}$.

On the other hand, if we know $\pi _{1}$ and $\pi _{2}$ then we could
construct a class of unbiasedness estimators (only) by setting
\begin{equation*}
a_{2}=\frac{1-a_{1}\pi _{1}}{\pi _{2}}\qquad \text{and}\qquad
a_{3}=a_{1}\left( 1-\pi _{1}\right) +a_{2}\left( 1-\pi _{2}\right)
\end{equation*}

\textit{(I realize that I am going off on a tangent here, but...) }We can
restate (\ref{mean2}) as
\begin{equation*}
E\left[ a_{1}\overline{X}_{1}+a_{2}\overline{X}_{2}\right] =\left( a_{1}\pi
_{1}+a_{2}\pi _{2}\right) \mu +\left( a_{1}\left( 1-\pi _{1}\right)
+a_{2}\left( 1-\pi _{2}\right) \right) \kappa
\end{equation*}%
or if we assume that $\pi _{1}=\pi _{2}$ and $a_{1}=a_{2}=\frac{1}{2}$ (the
scale of the $a$'s is irrelevant and at this point it seems meaningless to
distinguish between $\overline{X}_{1}$ and $\overline{X}_{2}$)
\begin{equation*}
E\left[ \overline{X}\right] -\left( 1-\pi \right) \kappa =\pi \mu
\end{equation*}%
or%
\begin{equation*}
\mu =\frac{E\left[ \overline{X}\right] -\left( 1-\pi \right) \kappa }{\pi }
\end{equation*}%
The derivative of this with respect to $\pi $ is
\begin{equation*}
\frac{\pi \kappa -E\left[ \overline{X}\right] +\left( 1-\pi \right) \kappa }{%
\pi ^{2}}=\frac{\kappa -E\left[ \overline{X}\right] }{\pi ^{2}}
\end{equation*}%
This is monotone, so if we know that $\pi \in \left[ \pi ^{o},1\right] $,
then we know that $\mu $ must be between $E\left[ \overline{X}\right] $ and $%
\frac{E\left[ \overline{X}\right] -\left( 1-\pi ^{o}\right) \kappa }{\pi ^{o}%
}$.

Note that

\begin{itemize}
\item Allowing different $a_{1}$ and $a_{2}$ should not provide more
information.

\item There seems to be no scope for identifying $\pi $
\end{itemize}

We next turn to the case where different observations have different number
of matches. In some (unrealistic) cases, this may sharpen the bound above...

Suppuse that for $\ell =1,..,L$, each of $X_{i\ell }$ has the distribution
of interest with probability $\pi _{L}/L$ (these are mutually exclusive).
Let
\begin{equation*}
\overline{X}=\frac{1}{n}\sum_{i=1}^{n}\left( X_{i1}+...+X_{iL}\right)
\end{equation*}%
the
\begin{equation}
E\left[ \overline{X}\right] =\pi _{L}\mu +\left( L-1\right) \kappa +\left(
1-\pi _{L}\right) \kappa  \label{meanL}
\end{equation}%
or%
\begin{equation*}
\mu =\frac{E\left[ \overline{X}\right] -\left( L-1\right) \kappa -\left(
1-\pi _{L}\right) \kappa }{\pi _{L}}
\end{equation*}%
This is monotone in $\pi _{L}$ so if we know that $\pi _{L}\in \left[ \pi
_{L}^{o},1\right] $ then $\mu $ must be between $E\left[ \overline{X}\right]
$ and $\frac{E\left[ \overline{X}\right] -\left( L-1\right) \kappa -\left(
1-\pi _{L}^{o}\right) \kappa }{\pi _{L}^{o}}$.

If we know how $\pi _{L}$ varies with $L$ then identification may be
possible. But not always. Suppose, for example that $\pi _{L\text{ }}$ is
constant, then (\ref{meanL}) with $L=2$ and $3$ yields%
\begin{eqnarray*}
E_{L=2}\left[ \overline{X}\right] &=&\pi \mu +\kappa +\left( 1-\pi \right)
\kappa \\
E_{L=3}\left[ \overline{X}\right] &=&\pi \mu +2\kappa +\left( 1-\pi \right)
\kappa
\end{eqnarray*}%
Thought of as functions of $\left( \pi \mu \right) $ and $\pi $ these are
two linear equations in two unknowns... But they are collinear...

\subsection{Linear IV}

We now return to the setup in section \ref{Lin IV}. Suppose that there are $%
L_{i}\ $matched $y$'s for $x_{i}$. There is probability $\pi _{i}$ that one
of them is teh correct match. The others are drawn from the distribution of $%
y$ connditional on $w_{i}$

Suppose we know $\pi _{i}$. We can then write%
\begin{equation}
\sum_{\ell }y_{i\ell }=\left( L_{i}-\pi _{i}\right) g\left( w_{i}\right)
+\pi _{i}x_{i}^{\prime }\beta +u_{i}
\end{equation}%
where $u_{i}$ is potentially correleted with $x_{i}\ $\ but uncorreleated
with the instrument $z_{i}$. We can rewrite to get%
\begin{equation}
\sum_{\ell }y_{i\ell }-L_{i}g\left( w_{i}\right) =-\pi _{i}g\left(
w_{i}\right) +\pi _{i}x_{i}^{\prime }\beta +u_{i}
\end{equation}%
or
\begin{equation*}
\gamma _{i}\sum_{\ell }y_{i\ell }-\gamma _{i}L_{i}g\left( w_{i}\right)
+g\left( w_{i}\right) =x_{i}^{\prime }\beta +\varepsilon _{i}
\end{equation*}%
where $\gamma _{i}=1/\pi _{i}$. 2sls applied to thsi yields%
\begin{multline*}
\left( \left( \dsum x_{i}z_{i}^{\prime }\right) \left( \dsum
z_{i}z_{i}^{\prime }\right) ^{-1}\left( \dsum x_{i}z_{i}^{\prime }\right)
^{\prime }\right) ^{-1}\left( \dsum x_{i}z_{i}^{\prime }\right) \left( \dsum
z_{i}z_{i}^{\prime }\right) ^{-1} \\
\left( \dsum z_{i}\left( \gamma _{i}\left( \sum_{\ell }y_{i\ell
}-L_{i}g\left( w_{i}\right) \right) +g\left( w_{i}\right) \right) \right)
\end{multline*}%
Let $M=\left( \left( \dsum x_{i}z_{i}^{\prime }\right) \left( \dsum
z_{i}z_{i}^{\prime }\right) ^{-1}\left( \dsum x_{i}z_{i}^{\prime }\right)
^{\prime }\right) ^{-1}\left( \dsum x_{i}z_{i}^{\prime }\right) \left( \dsum
z_{i}z_{i}^{\prime }\right) ^{-1}$ and $m_{i}=Mz_{i}$ (a column vector) the
the 2SLS estimator is%
\begin{equation*}
\dsum_{i}\gamma _{i}m_{i}\left( \sum_{\ell }y_{i\ell }-L_{i}g\left(
w_{i}\right) \right) +\dsum_{i}m_{i}g\left( w_{i}\right)
\end{equation*}%
Suppose we are interested in the first first element of $\widehat{\beta }$.
It can be written as
\begin{equation*}
\widehat{\beta }_{1}=\dsum_{i}\gamma _{i}m_{1i}\left( \sum_{\ell }y_{i\ell
}-L_{i}g\left( w_{i}\right) \right) +\dsum_{i}m_{1i}g\left( w_{i}\right)
\end{equation*}%
where $m_{1i}$ is the first element of. Or%
\begin{equation*}
\widehat{\beta }_{1}=\dsum_{i}\gamma _{i}a_{1i}+a_{1}
\end{equation*}%
where $a_{1i}=m_{1i}\left( \sum_{\ell }y_{i\ell }-L_{i}g\left( w_{i}\right)
\right) $ and $a_{1}=\dsum_{i}m_{1i}g\left( w_{i}\right) $

Of course, in practive we dont actually know $\pi $ (or $\gamma =\pi ^{-1}$)
except that we know that $\underline{\pi }\leq \pi \leq 1$ and hence $1\leq
\gamma \leq \overline{\gamma }=\underline{\pi }^{-1}$. Then the lower and
upper bounds for $\widehat{\beta }_{1}$ are obtained by
\begin{equation*}
\left[ \dsum_{i}1\left\{ a_{1i}>0\right\} a_{1i}+\dsum_{i}\overline{\gamma }%
1\left\{ a_{1i}<0\right\} a_{1i}+a_{1},\dsum_{i}1\left\{ a_{1i}<0\right\}
a_{1i}+\dsum_{i}\overline{\gamma }1\left\{ a_{1i}>0\right\} a_{1i}+a_{1}%
\right]
\end{equation*}%
This is the sample analog of the set of 2SLS\ estimands generated by all
possible assignments $\pi _{i}$

\subsection{Tradeoff between interval size and estimation precision when
choosing to discard low-probability matches}

\subsection{Maximum Likelihood}

\section{Application}

In this section we use data from \cite{AizerEliFerrieLLerasMuney2016} to
illustrate. \footnote{%
The data, mp\_data.dta is posted as part of the publication:
https://www.aeaweb.org/articles?id=10.1257\%2Faer.20140529, and it provides
multiple matches for a subset of the observations.}

\section{Conclusion}

% This paper is not related to one of the few interesting papers using bounds, \cite{HonoreLleras_Muney06}.

\bibliographystyle{econometrica}
\bibliography{bos_ref}

\end{document}
