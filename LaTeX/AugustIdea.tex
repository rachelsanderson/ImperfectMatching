\documentclass[11pt]{amsart}
\usepackage{geometry}                % See geometry.pdf to learn the layout options. There are lots.
\geometry{letterpaper}                   % ... or a4paper or a5paper or ... 
%\geometry{landscape}                % Activate for for rotated page geometry
%\usepackage[parfill]{parskip}    % Activate to begin paragraphs with an empty line rather than an indent
\usepackage{graphicx}
\usepackage{amssymb}
\usepackage{epstopdf}
\DeclareGraphicsRule{.tif}{png}{.png}{`convert #1 `dirname #1`/`basename #1 .tif`.png}

\newcommand{\toP}{\overset{p}{\to}}
\newcommand{\meanN}{\frac{1}{n}\sum_{i=1}^n}
\newcommand{\sumL}{\sum_{\ell=1}^{L_i}}
\newcommand{\meanL}{\frac{1}{L_i}\sumL}
\newcommand{\sumJ}{\sum_{j=1}^{J_i}}

\title{Research Ideas}
\author{Rachel Anderson}
\date{\today}                                           % Activate to display a given date or no date

\begin{document}
\maketitle
%\section{}
%\subsection{}


\begin{enumerate} 
\item What do we gain by using observations with multiple matches?  Are there cases where we are better off excluding them in finite sample inference? (Bias/variance tradeoff)

\item Can we replicate the results from AER using a fully Bayesian framework?     

 \end{enumerate}

Consider the case where the number of individuals with more than one match, $N_{L}$ is small relative to the population, of size $N$ (like the AER application where about 4\% of observations have multiple linked outcomes.)  It may then be feasible to solve:

\begin{gather*}\hat{\beta}^{\min/\max} = \min/\max_{\{w_{i\ell}\}} \left(\meaN x_ix_i'\right)^{-1} \left(\meanN x_i \overline{y}_i \right) \\
\overline{y}_i = \sumL w_{i\ell} y_{i\ell} \\
\sumL w_{i\ell} = 1, \ w_{i\ell} \geq 0  \end{gather*}


Also, we can try alternating $\sumL w_{i\ell} = 1$ OR $\sumL w_{i\ell} = 0$ for observations with $L_i>1$ only, to compare whether they should be included in analysis or not.   


Compare this to our method, which uses the moment condtion:

$$m(y_{i\ell}, x_i; \theta) = x_i(y_{i\ell}-x_i'\beta)  $$ 

so that the estimator uses

\begin{gather*} \meanN \sumL   x_i(y_{i\ell}-x_i'\beta)   - \meanN (L_i-1) \hat{g}(w_i, L_i, x_i; \beta) \\
 g(w_i, L_i, x_i; \beta) = x_i E(y_{i\ell} | w_i, L_i) - x_ix_i'\beta
\end{gather*} 

Rewriting,
\begin{gather*} \hat{\beta} = (X'X)^{-1}\meanN \left( \sumL x_i y_{i\ell} - (L_i -1) x_i E[y | w_i]\right) \\
= (X'X)^{-1} \meanN \left( \sum_{L_i = 1} x_i y_i + \sum_{L_i > 1}  x_i \left(E[y | w_i] + \sumL (y_{i\ell} - E[y | w_i] ) \right)\right) \\
= (X'X)^{-1} \frac{N - N_L}{N} \frac{1}{N-N_L} \sum_{L_i = 1} x_i y_i + (X'X)^{-1} \frac{N_L}{N}\frac{1}{N_L} \sum_{L_i> 1} x_i \left(E[y|w_i] + \sum_{\ell=1}^{L_i}(y_{i\ell} - E[y  | w_i])\right)\\ 
%\toP E[x_ix_i']^{-1} (1-\lambda) E[x_iy_i | L_i = 1] +  E[x_ix_i']^{-1} \lambda \left(E[x_i E[y|w]] + E[x_i \sumL y_{i\ell} - E[y | w_i] ]\right)
\toP \beta
\end{gather*}
by the theorem


but what if we use the first term only?  Or use a different kernel on the second term in order to change variance?  We could get bias/variance tradeoff

Compare to unbalanced panel techniques/ random effects estimators 
\end{document}  